\documentclass[a4paper,11pt]{exam}
\usepackage[german, english]{babel}
\usepackage[applemac]{inputenc}

\pointpoints{\%}{\%}
\begin{document}
\begin{center} \fbox{\fbox{\parbox{5.5in}{\centering
Answer the questions in the spaces provided on the question sheets. If you run out of room for an answer, continue on the back of the page.}}}
\end{center}\makebox[\textwidth]{Name: \enspace\hrulefill}

\begin{questions}
\question[30] {\bf Handling of an Ordering}

A customer orders a product. The selling department handles the order, which is then followed by the collection of goods in stock
and the dispatch. While the goods are on their way to the customer, the department of invoicing is creating the invoice. The invoice
is paid as soon as the goods and the invoice arrive at the customer. Finally the invoice will be closed.

Make a UML activity diagram to describe the above. Take care of the areas of responsibility.

\question{\bf ATM}
A simplified ATM transaction could be described as follows:

\begin{itemize}
	\item The customer inserts the bank card
	\item The customer enters his/her pin code
	\item The customer enters the amount of money she/he wants to withdraw
	\item The ATM checks the PIN
	\item Parallel to this the ATM checks the liquidity of the customer at his/her home bank
	\item If the customer entered the correct PIN code and his/her transaction is covered by his/her account, (s)he is returned the bank card and then (s)he gets cash.
	\item Otherwise an error message is shown and the customer gets the bank card.
\end{itemize}
Model this behavior using a UML activity diagram.

\begin{otherlanguage}{german}
Vereinfacht stellt sich das Geldabheben an einem Automaten wie folgt dar:

\begin{itemize}
	\item Der Kunde schiebt die Bankomatkarte in den Bankomaten
	\item Der Kunde gibt dem Geldautomaten seine Geheimzahl ein
	\item Der Kunde gibt an, wie viel Geld er abheben m�chte
	\item Der Geldautomat �berpr�ft die Geheimzahl
	\item Gleichzeitig �berpr�ft der Automat die Liquidit�t des Kunden bei dessen Bank
	\item Wenn der Kunde liquide ist und die korrekte Geheimzahl eingegeben hat, erh�lt der nacheinander die Karte und das Bargeld vom Bankomaten.
	\item Andernfalls erh�lt der Kunde eine Fehlermeldung angezeigt und danach die Karte zur�ck
\end{itemize}
Beschreiben Sie diesen Sachverhalt mittels eines UML Aktivit�tsdiagramms.
\end{otherlanguage}

\end{questions}
\end{document}