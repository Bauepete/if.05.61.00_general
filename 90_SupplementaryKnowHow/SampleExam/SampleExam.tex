\documentclass[a4paper,11pt]{exam}

\pointpoints{per cent}{per cents}
\begin{document}
\begin{center} \fbox{\fbox{\parbox{5.5in}{\centering
Answer the questions in the spaces provided on the question sheets. If you run out of room for an answer, continue on the back of the page.}}}
\end{center}\makebox[\textwidth]{Name: \enspace\hrulefill}
\begin{questions}
\question[70] {\bf Museum}

A managing director of a museum asks you to prepare a data model for the following description:

The artworks exhibited in the museum are categorized into
\begin{description}
	\item[Kind of artwork] Painting, Sculpture, etc.
	\item [Style] Modern art, Art Nouveau, etc.
\end{description}

For every artwork there is exactly one artist who created it. The artworks are located at exactly one position in the museum but
can be moved to other locations too. Artworks can be lended to other museums which have to apply for this in advance. The
origin of the artworks is also documented by maintaining a list of previous owners who are stored into a separate table.

By means of the data model the following questions have to be answered:
\begin{itemize}
	\item Which kind of artworks does an artist create?
	\item Which previous owner does an artwork have?
	\item At which locations of the museum was the artwork already exhibited and how long?
	\item Sum of the purchase price of the artworks of style "Art Nouveau".
	\item All paintings which are lended more than 10 times a year.
	\item Total duration of lending per kind
	\item Are all the artworks which were applied for lending really lended?
\end{itemize}
\begin{parts}
	\part Create a class diagram to describe the facts given above
	\part Give all attributes for the classes
\end{parts}
If you miss some more specific detail make assumptions and document them.

\question[30] {\bf Handling of an Ordering}

A customer orders a product. Die selling department handles the order, which is then followed by the collection of goods in stock
and the dispatch. While the goods are on their way to the customer, the department of invoicing is creating the invoice. The invoice
is paid as soon as the goods and the invoice arrive at the customer. Finally the invoice will be closed.

Make a UML activity diagram to describe the above. Take care of the areas of responsibility.
\end{questions}
\end{document}